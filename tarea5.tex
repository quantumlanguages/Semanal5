% document type
\documentclass{article}

% page format
\usepackage[letterpaper, margin = 1.5cm]{geometry}

% header
\title{
    Lenguajes de Programación 2020-1 \\
    Facultad de Ciencias UNAM \\
    Ejercicio Semanal 5
}

\author{
    Sandra del Mar Soto Corderi \\
    Edgar Quiroz Castañeda
}

\date{
    12 de septiembre de 2019
}

\begin{document}
    \maketitle
    \begin{enumerate}
        \item {
            De acuerdo a la representación de lo booleanos en el cálculo lambda,
            realice lo siguiente
            \begin{itemize}
                \item {
                    Implemente la función de disyunción \texttt{or}
                }
                \item {
                    Implemente la función de disyunción exclusiva \texttt{xor}
                }
                \item {
                    Muestre que las definiciones son correctas. Esto es que para
                    $b_1, b_2$ boolenanos y \texttt{f} función booleana,
                    \texttt{f} $b_1, b_2$ se recuden de acuerdo a la tabla de
                    verdad de \texttt{f}. 
                }
            \end{itemize}
        }

        \item {
            Encientre la forma normal de la siguiente expresión $e$. Se debe
            mostrar todos los pasos de la $\beta-$reducción, indicando en cada
            paso el redex que se va a reducir.

            \[
                e =_{def} (\lambda x. \lambda y. \lambda z.xz(yz))
                (\lambda s. \lambda t.s)
                (\lambda u. u) w
            \]
        }
    \end{enumerate}
\end{document}