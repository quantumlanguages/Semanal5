% document type
\documentclass{article}

% page format
\usepackage[letterpaper, margin = 1.5cm]{geometry}

% environment placement
\usepackage{float}

% math things
\usepackage{amsmath}

% header
\title{
    Lenguajes de Programación 2020-1 \\
    Facultad de Ciencias UNAM \\
    Ejercicio Semanal 5
}

\author{
    Sandra del Mar Soto Corderi \\
    Edgar Quiroz Castañeda
}

\date{
    12 de septiembre de 2019
}

\begin{document}
    \maketitle
    \begin{enumerate}
        \item {
            De acuerdo a la representación de lo booleanos en el cálculo lambda,
            realice lo siguiente
            \begin{itemize}
                \item {
                    Implemente la función de disyunción \texttt{or}

                    \[
                        \texttt{or} \equiv 
                        \lambda p. \lambda q. p T q
                        \equiv \lambda p. \lambda q. p (\lambda x. \lambda y. x) q 
                    \]
                }
                \item {
                    Implemente la función de disyunción exclusiva \texttt{xor}

                    \[
                        \texttt{xor} \equiv 
                        \lambda p. \lambda q. p (q F T) (q T F)
                        \equiv \lambda p. \lambda q. p 
                        (q (\lambda x. \lambda y. y) (\lambda x. \lambda y. x)) 
                        (q (\lambda x. \lambda y. x) (\lambda x. \lambda y. y))
                    \]
                }
                \item {
                    Muestre que las definiciones son correctas. Esto es que para
                    $b_1, b_2$ boolenanos y \texttt{f} función booleana,
                    \texttt{f} $b_1, b_2$ se reducen de acuerdo a la tabla de
                    verdad de \texttt{f}.

                    Notemos que en todas las apariciones de $T$ y $F$, se puede
                    suponer que no tiene variables en común con ninguna otra
                    función en la expresión, pues no están expresada de forma
                    explícita.
                    \begin{itemize}
                        \item {
                            \texttt{or}

                            La table de verdad es 

                            \begin{table}[H]
                                \centering
                                \begin{tabular}{|l|l|l|}
                                    \hline
                                    $\lor$ & $F$ & $T$ \\ \hline
                                    $F$    & $F$ & $T$ \\ \hline
                                    $T$    & $T$ & $T$ \\ \hline
                                \end{tabular}
                                \caption{Tabla de verdad de $p \lor q$}
                                \label{tab:truth_or}
                            \end{table}

                            Entonces, primero supongamos que 
                            $P \equiv F$.

                            Entonces, 
                            \begin{align*}
                                \texttt{or} F Q &\equiv F T Q
                                \equiv (\lambda x. \lambda y. y)
                                (\lambda x. \lambda y. x)
                                Q \\
                                &\equiv (\lambda y. y)
                                [x:=\lambda x. \lambda y. x]
                                Q
                                =(\lambda y. y[x:=\lambda x. \lambda y. x])
                                (\lambda x. \lambda y. y) Q \\
                                &= (\lambda y. y) Q \equiv y[y := Q] = Q
                            \end{align*}

                            Por lo que $\texttt{or}FF = F$ y $\texttt{or}FT = T$
                            , lo que corresponde a la primera columna de la
                            tabla de verdad.

                            En otro caso, tomemos a $P = T$.

                            Entonces
                            \begin{align*}
                                \texttt{or} T Q &\equiv T T Q
                                \equiv (\lambda x. \lambda y. x)
                                (\lambda w. \lambda z. w)
                                Q \\
                                &\equiv (\lambda y. x)
                                [x:=\lambda w. \lambda z. w]
                                Q
                                =(\lambda y. \lambda w. \lambda z. w) Q \\
                                &\equiv (\lambda w. \lambda z. w) [y := Q] 
                                = \lambda w. \lambda z. w \equiv T
                            \end{align*}

                            Por lo que $\texttt{or}TT = T$ y $\texttt{or}TT = T$,
                            que corresponde con la segunda columna de la tabla
                            de verdad.

                            Como la función corresponde en todos los casos con
                            la tabla de verdad, entonces la definición de la 
                            función es correcta.
                        }
                        \item {
                            $\texttt{xor}$

                            La tabla de verdad es 

                            \begin{table}[H]
                                \centering
                                \begin{tabular}{|l|l|l|}
                                    \hline
                                    $\oplus$ & $F$ & $T$ \\ \hline
                                    $F$      & $F$ & $T$ \\ \hline
                                    $T$      & $T$ & $F$ \\ \hline
                                \end{tabular}
                                \caption{Tabla de verdad de $p \oplus q$}
                                \label{tab:truth_xor}
                            \end{table}

                            Primero, tomemos a $P = T$.

                            \begin{align*}
                                \texttt{xor} T Q &\equiv T(QFT)(QTF) \\
                                &\equiv (\lambda x. \lambda y. x) (QFT)(QTF) \\
                                &\equiv (\lambda y. x)[x := QFT](QTF) \\
                                &\equiv (\lambda y. QFT)(QTF) \\
                                &\equiv QFT[y := QTF] \equiv QFT
                            \end{align*}

                            En otro caso,$P = F$

                            \begin{align*}
                                \texttt{xor} F Q &\equiv F(QFT)(QTF) \\
                                &\equiv (\lambda x. \lambda y. y) (QFT)(QTF) \\
                                &\equiv (\lambda y. y)[x := QFT](QTF) \\
                                &\equiv (\lambda y. y)(QTF) \\
                                &\equiv y[y := QTF] \equiv QTF
                            \end{align*}

                            Sin pérdidad de generalidad, digamos que en ambos
                            casos la expresión resultante es $QMN$.

                            Luego, supongamos que $Q = T$. 

                            Entonces
                            \begin{align*}
                                TMN &\equiv (\lambda x. \lambda y. x) MN \\
                                &\equiv ((\lambda y. x)[x := M]) N \\
                                &\equiv (\lambda y. (x[x := M])) N \\
                                &\equiv (\lambda y. M) N \equiv M[y := N] \\
                                &\equiv M
                            \end{align*}

                            Entonces, $P = T \implies M = F$, por lo que
                            $\texttt{xor} T T = F$, que corresponde con la tabla
                            de verdad.

                            Por otra parte, $P = F \implies M = T$, por lo que
                            $\texttt{xor} F T = F$, que corresponde con la tabla
                            de verdad.

                            Luego, veamos que pasa cuando $Q = F$.

                            \begin{align*}
                                FMN &\equiv (\lambda x. \lambda y. y) MN \\
                                &\equiv ((\lambda y. y)[x := M]) N \\
                                &\equiv (\lambda y. (y[x := M])) N \\
                                &\equiv (\lambda y. y) N \equiv y[y := N] \\
                                &\equiv N
                            \end{align*}

                            Entonces, $P = T \implies N = T$, por lo que
                            $\texttt{xor} T F = T$, que corresponde con la tabla
                            de verdad.

                            Por otra parte, $P = F \implies N = F$, por lo que
                            $\texttt{xor} F F = F$, que corresponde con la tabla
                            de verdad.

                            Como en todos los casos se obtiene el resultado dado
                            por la tabla de verdad, entonces la definición de la
                            función es correcta.
                        }
                    \end{itemize}
                }
            \end{itemize}
        }

        \item {
            Encuentre la forma normal de la siguiente expresión $e$. Se debe
            mostrar todos los pasos de la $\beta-$reducción, indicando en cada
            paso el redex que se va a reducir.

            \[
                e =_{def} (\lambda x. \lambda y. \lambda z.xz(yz))
                (\lambda s. \lambda t.s)
                (\lambda u. u) w
            \]

            \begin{align*}
                e &\equiv (\lambda y. \lambda z.xz(yz))[x:=(\lambda s. \lambda t.s)]
                (\lambda u. u) w \\
                &\equiv (\lambda y. (\lambda z.xz(yz))[x:=(\lambda s. \lambda t.s)]
                (\lambda u. u) w \\
                &\equiv \lambda y. \lambda z.(xz(yz))[x:=(\lambda s. \lambda t.s)])
                (\lambda u. u) w \\
                &\equiv (\lambda y. \lambda z.
                x[x:=(\lambda s. \lambda t.s)]
                z[x:=(\lambda s. \lambda t.s)]
                (yz)[x:=(\lambda s. \lambda t.s)])
                (\lambda u. u) w \\
                &\equiv (\lambda y. \lambda z.
                (\lambda s. \lambda t.s)z(yz))
                (\lambda u. u) w \\
                &\equiv (\lambda y. \lambda z.
                (\lambda t.s)[s := z](yz))
                (\lambda u. u) w \\
                &\equiv (\lambda y. \lambda z.
                \lambda t.(s[s := z])(yz))
                (\lambda u. u) w \\
                &\equiv (\lambda y. \lambda z.
                (\lambda t.z)(yz))
                (\lambda u. u) w \\
                &\equiv (\lambda y. \lambda z.
                (z[t:=yz]))
                (\lambda u. u) w \\
                &\equiv (\lambda y. \lambda z.z)
                (\lambda u. u) w \\
                &\equiv (\lambda z.z) [y:=\lambda u. u] w \\
                &\equiv (\lambda z.z[y:=\lambda u. u]) w \\
                &\equiv (\lambda z.z) w \\
                &\equiv z [z:=w] \\
                &\equiv w \\
            \end{align*}
        }
    \end{enumerate}
\end{document}